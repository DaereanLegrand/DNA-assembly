\documentclass{article}
\usepackage[spanish]{babel}
\usepackage{hyperref}
\usepackage{graphicx}

\hypersetup{
    colorlinks=true,
    linkcolor=blue,
    filecolor=blue,      
    urlcolor=blue,
    pdftitle={Overleaf Example},
    pdfpagemode=FullScreen,
    }

\title{Laboratorio 03a: Ensamblaje de Fragmentos de ADN}
\author{Frank Roger Salas Ticona}
\date{\today}

\begin{document}
\maketitle

\section{Introducción}
En el campo de la biología molecular computacional, el ensamblaje de fragmentos de ADN es una tarea fundamental que permite reconstruir secuencias genéticas completas a partir de lecturas cortas. Este informe presenta la implementación y análisis de un algoritmo de ensamblaje de ADN basado en la búsqueda de caminos hamiltonianos en grafos. Utilizando un enfoque que combina programación en C para el procesamiento eficiente de datos y Python para la visualización de resultados, nuestro método aborda el desafío de encontrar la secuencia de consenso óptima y visualizar las relaciones entre los fragmentos de ADN. A lo largo de este informe, se detallarán los aspectos técnicos de la implementación, se analizarán los resultados obtenidos y se discutirán las implicaciones y posibles mejoras de esta aproximación al problema del ensamblaje de ADN.
\section{Implementación}
\section{Resultados}
\subsection{Experimentos}
\begin{figure}[!htbp]
    \centering
    %\includegraphics[width=0.5\textwidth]{}
    \caption{Algoritmo UPGMA corriendo con las secuencias de ejemplo.}
    \label{fig:code1}
\end{figure}
\section{Análisis y Discusión}
\section{Conclusiones}

\end{document}
